%
% Title page layout for submitted version.
%
\newcommand*{\forcetwosidetitle}[1][1]{%
 \begingroup
   \cleardoubleoddpage
   % \KOMAoptions{titlepage=true}% useful e.g. for scrartcl
   \csname @twosidetrue\endcsname
   \maketitle[{#1}]
 \endgroup
}

\title{\thesistitle}
\subtitle{\vspace*{4ex}
  \begin{otherlanguage}{ngerman}
    Dissertation\\
    zur\\
    Erlangung des Doktorgrades (Dr.\ rer.\ nat.)\\
    der\\
    Mathematisch-Naturwissenschaftlichen Fakultät\\
    der\\
    Rheinischen Friedrich-Wilhelms-Universität Bonn
  \end{otherlanguage}
}
\author{%
  vorgelegt von\\
  \thesisauthor\\
  aus\\
  \thesistown
}
\date{}
\publishers{%
  Bonn \thesisyear
}
\lowertitleback{\normalsize
  \begin{otherlanguage}{ngerman}
    Angefertigt mit Genehmigung der
    Mathematisch-Naturwissenschaftlichen Fakultät der
    Rheinischen Friedrich-Wilhelms-Universität Bonn
  \end{otherlanguage}

  \vspace*{10ex minus 4ex}

  \noindent
  \begin{otherlanguage}{ngerman}
    \begin{tabular}{@{}ll}
      \thesisrefereeonetext: & \thesisrefereeone\\
      \thesisrefereetwotext: & \thesisrefereetwo\\[2ex]
      Tag der Promotion:      & \\
      Erscheinungsjahr:       &
    \end{tabular}
  \end{otherlanguage}
}

\forcetwosidetitle
